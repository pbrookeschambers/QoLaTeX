\documentclass{article}

\usepackage[margin = 0.7in]{geometry}

\usepackage[scheme = twilight, light]{../QoColours/qocolours}
\usepackage[frame = tlr, background]{./qolistings}
\usepackage{hyperref}
\usepackage{enumitem}
\usepackage{amsmath, amssymb}
\usepackage{lipsum}
\usepackage{multicol}
\usetikzlibrary{positioning}
\setlength\parskip{1.5em}

\title{\texttt{qolistings.sty} Package}
\author{Peter Brookes Chambers}
\date{\today{}}
% \newcommand{\option}[3]{\begin{multicols}{3}\noindent\texttt{\color{Accent1}#1}\leavevmode\\\textit{#2}\leavevmode\\\textit{#3}\end{multicols}\leavevmode}
\newcommand{\option}[4][]{\noindent\begin{tikzpicture}
\node[right] (option) at (0, 0) {\texttt{\color{Accent1}#2}};
\node[] at (\linewidth * 0.5, 0) {\normalcolor\textit{#3}};
\node[left] at (\linewidth, 0) {\normalcolor\textit{#4}};
\node[below = 0.0em of option.south west, anchor = north west] {\color{Accent2}\texttt{#1}\normalcolor};
\end{tikzpicture}
}
% \newcommand{\option}[3]{\noindent\texttt{\color{Accent1}#1}\hfill\textit{#2}\hfill\textit{#3}}

\begin{document}

\pagenumbering{gobble}
\maketitle
\clearpage
\tableofcontents{}
\clearpage
\pagenumbering{arabic}
\section{Introduction}

This package builds on the \href{https://ctan.math.illinois.edu/macros/latex/contrib/listings/listings.pdf}{\texttt{listings}} package to provide apealing and professional default styles with an easy interface for common options.

\section{Useage}

\subsection{The \texttt{\textbackslash{}qoinputlisting} Command}

This command is analogous to the \verb|\lstinputlisting| command provided by the \texttt{listings} package. It takes one mandatory argument, which should be the name of the file to include, and one optional argument which should be a series of comma separated key-value pairs (parsed by the \texttt{keyval} package). These key-value pairs are as follows, where the option is given on the left, the possible values in the centre (a ``-" indicates that the option will default to ``true" if no value is given), and the default value is given on the right.

\subsection{Optional Arguments}

The optional arguments for the commands in the \verb|\qolistings| package are listed here in \color{Accent1}yellow\normalcolor{}, followed by the allowable values, then the default value. A value of ``-" indicates that the key can be given with no values, in which case it is given a value of ``true". The corresponding global options, where appropriate, are given in \color{Accent2}blue\normalcolor{}.

\option{lst options}{any}{\{\}}

This option takes as its argument a list of key-value pairs to be passed to the \texttt{listings} package, which should be encased in single braces only (for example, \verb|lst options = {stringstyle = \color{red}, commentstyle = \tiny}|). Encasing these in double braces will cause errors as this will not be correctly expanded before being passed to the \texttt{keyval} package by \texttt{listings}.

These key-value pairs must also be valid options for a style in the \texttt{listings} package. These are applied \textbf{after} the default style (and any other options), and so if a key is given a value in both \texttt{lst options} and the default style, the value given in \texttt{lst options} is used.

\option{lst early options}{any}{\{\}}

This option is almost identical to the \texttt{lst options} option, except that it is applied \textbf{before} the default style (and subsequently before \texttt{lst options}), and so key-value pairs given in \texttt{lst early options} have the lowest priority. This doesn't currently have many direct applications, but should allow for more compatability should this package be expanded.

\option[latexcomments]{latex comments}{-, true, false}{false}

If true, then comments encased by the \texttt{escapeinside} tokens are rendered as normal \LaTeX{} code rather than printed verbatim. By default, the escape tokens are ``\verb|(*|" and ``\verb|*)|", though these can of course be changed by passing \verb|{escapeinside = {<token1>}{<token2>}}| to the \texttt{lst options} key.

\option[latexmaths]{latex maths}{-, true, false}{false}\\
\option[latexmath]{latex math}{-, true, false}{false}

If true (either with the U.S. or U.K. spelling), anything encased in dollar signs will be rendered as \LaTeX{} maths rather than printed verbatim. Note that this applies to the entire file, not just to comments, and is independent of the \texttt{latex comments} option. This means that any \$ anywhere in the file must be accompanied by a corresponding closing \$, and the contents \textbf{must} be valid input for a \LaTeX{} maths environment. (Be especially careful of escape sequences! Even strings written to be valid \LaTeX{} when printed or passed to plotting tools may not be if backslashes need to be escaped.)

\option[latex]{latex}{-, true, false}{}

This flag simply sets both \texttt{latex comments} and \texttt{latex maths} to its own value. This is processed before the \texttt{latex comments} and \texttt{latex maths} options, so their values will overridde the \texttt{latex} flag.


\option{mdframed options}{any}{\{\}}

For better frame and background handling, all listings are encased in an \texttt{mdframed} environment. This option takes as its argument a list of key-value pairs to be passed to the \texttt{mdframed} package. Again, these should be encased in a single brace only.

These key-value pairs must also be valid options for a style in the \texttt{mdframed} package. These are applied last, after any other key-value pairs are handled for the \texttt{mdframed} environment.

\option[background]{background}{-, true, false}{false}

If true, the listing is given a background colour of \texttt{BackgroundColour}.


\option[roundedcorners]{rounded corners}{-, true, false}{false}

If true, the listing is given rounded corners with radius 0.5\,em. This has no effect if both \texttt{background} and \texttt{frame} are false.

\option[frame]{frame}{-, true, false, ltrb}{false}

This option handles the frame of the listing. If true, then the listing is given a frame on all four sides. This option can also take any combination of ``\texttt{l}", ``\texttt{t}", ``\texttt{r}", and ``\texttt{b}", which correspond to a left, top, right, and bottom edge frame. Any combination of these (in any order) will draw a frame on each edge given. For example, \texttt{frame = blr} will draw a frame on all but the top edge.

\end{document}
