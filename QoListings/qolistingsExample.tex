\documentclass{article}

\usepackage[margin = 0.7in]{geometry}

\usepackage{../officecolours/officecolours}
\usepackage{./qolistings}
\usepackage{hyperref}
\usepackage{enumitem}
\usepackage{amsmath, amssymb}
\usepackage{lipsum}
\setlength\parskip{1.5em}

\title{\texttt{qolistings.sty} Package}
\author{Peter Brookes Chambers}
\date{\today{}}

\begin{document}

\pagenumbering{gobble}
\maketitle
\clearpage
\tableofcontents{}
\clearpage
\pagenumbering{arabic}
\section{Introduction}

This package defines and sets up several styles for the \texttt{listings} package, as well as a few useful supplimentary commands. This package requires four other packages: \texttt{listings}, \texttt{mdframed}, \texttt{xcolor}, and \texttt{forarray}. This package does not provide much functionality, and mostly exists for convenience. Each style uses specifically named colours, namely \texttt{ForegroundColour}, \texttt{ForegroundColour\_3}, \texttt{ForegroundColour\_5}, \texttt{BackgroundColour}, \texttt{Accent1}, and \texttt{Accent2}. As such, loading the \texttt{officecolours} package (or changing the scheme used by the \texttt{officecolours} package) will change the colours used in these styles. Thus, code listings should always fit in with the rest of a document.

However, the \texttt{officecolours} package is not necessary for the \texttt{qolistings} package, and so some default colours are defined and used \textbf{if and only if} no colour already has the names listed above. To this end, a new command is also defined; \verb|\colorprovide|. This acts identically to the \verb|\colorlet| command provided in the \texttt{xcolor} package, except that it will not overwrite an existing color. For example, \verb|\colorlet{blue}{red}| will overwrite the colour \texttt{blue} with the value of \texttt{red}. However, \verb|\colorprovide{blue}{red}| will change nothing, since a colour with the name \texttt{blue} already exists. In other words, \verb|\colorprovide| is to \verb|colorlet| as \verb|\providecolor| is to \verb|\definecolor|.

\section{Styles}

\subsection{\texttt{main}}
The first style defined is \texttt{main}, on which the subsequent styles are mostly based.

\lstinputlisting[style=latexcomments, language = python]{./exampleCode.py}
\framedinputlisting[style=mathcomments, language = python]{./exampleCode.py}

\lipsum[1-2]
\end{document}
